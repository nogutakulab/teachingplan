\documentclass[a4paper,11pt,dvipdfmx,uplatex,%
ja=standard]{bxjsarticle} 
%
\setpagelayout{margin=25mm} % 余白を25mmに
%%%%%%%%%%%%%%%%%=======プリアンブル==================%%%%%%%%%%
%
%%%%%%========パッケージの取り込み===========%%%%%%
%
\usepackage[T1]{fontenc} % フォントのあれこれ
\usepackage{lmodern}
\usepackage{bxpapersize} % 読み込みだけでOK
\usepackage{amsmath,amssymb} % 数式関連
\usepackage{ascmac} % 枠囲み
\usepackage{longtable} % ページをまたぐ長い表
\usepackage{ltablex}
\usepackage{braket} % カッコ
\usepackage{enumitem} % 箇条書き
\usepackage{otf} % 丸数字
%
\usepackage{./sty/teachingpractice} 
% タイトル形式の再定義
% 教育実習用でない場合は上を
%"./sty/teachingplan"に変える
%
%
%
\pagestyle{empty} % ページ番号を消去
%
\title{物理基礎 学習指導案} % タイトル
\date{2018年10月15日(月)} % 日付 
\period{1}{8:50}{10:20} 
% 時限,開始時刻,終了時刻の順に入力
\subject{2}{A}{38}
% 学年,クラス,人数の順に入力
\school{高知工科大学} % 学校名
\boss{NOGUTAKULab} % 指導教官
% \bossは"teachingplan.sty"のときは使えない
\author{野口 匠} % 実習生
% "teachingplan.sty"の場合は"授業者"になる
\venue{C306教室}
%
%%%%%%%%%=========ここから本文============%%%%%%%%%
\begin{document}
%
\maketitle
%
\section{単元名}
運動の法則

\section{使用教材}
いろいろ

\section{単元について}
\begin{enumerate}[label=(\arabic*)]
  \setlength{\parindent}{1zw}
  \item 単元観

    \verb|\setlength|コマンドにより,
    \verb|enumerate|環境下でもインデントができます。
    インデントをするためにはソース内に空行
    を入れてください。

    こんな感じ。

   

  \item 生徒観

    ここに生徒の様子を書く。


  \item 指導観

    ここにどう指導していくかを書く。
\end{enumerate}

\section{単元の目標}
ここに単元の目標を書く。

\section{単元の評価規準}
\begin{longtable}{|p{10zw}|p{10zw}|p{10zw}|p{10zw}|}
  % 10文字で改行
  \hline
  \multicolumn{1}{|c|}%
  {関心・意欲・態度} &
  \multicolumn{1}{|c|}%
  {思考・判断・表現} &
  \multicolumn{1}{|c|}%
  {観察・実験の技能} & 
  \multicolumn{1}{|c|}%
  {知識・理解}
  \\
  \hline
  \endfirsthead
  ・中点「・」は\verb|itemize|環境ではなく
  普通に・を打ちましょう。
  &
  ・hogehoge
  &
  ・hogehoge
  &
  ・hogehoge
  \\
  \hline
\end{longtable}

\section{単元における指導と評価の計画(全4時間)}
\begin{longtable}{|c|p{8zw}|p{22zw}|p{6zw}|}
  \hline
  \multicolumn{1}{|c|}{時数} &
  \multicolumn{1}{c|}{授業内容} &
  \multicolumn{1}{c|}{評価規準} &
  \multicolumn{1}{c|}{評価方法}
  \\
  \hline
  \endhead
  %
  1
  &
  タイトル
  &
  評価規準
  &
  評価方法
  \\
  &
  
  &
  評価規準その2
  &
  評価方法その2
  \\
  &

  &
  \verb|endhead|とあるところを
  \verb|endfirsthead|に変更すると,
  2ページ目以降のヘッダーが表示されなくなります。
  &
  その3
  \\
  \hline
  %
  2
  &
  タイトル2
  &
  評価規準  
  &
  評価方法
  \\
  &

  &
  評価規準2
  &
  評価方法2
  \\
  \hline
  %
  3
  &
  タイトル3
  &
  hogehoge
  &
  
  \\
  \hline
  %
  4(本時)
  &
  hogehoge
  &
  hogehoge
  &
  \\
  &

  &
  hogehoge
  &
  hogehoge
  \\
  \hline
  %
\end{longtable}

\section{本時の指導}
\begin{enumerate}[label=(\arabic*)]
  \setlength{\parindent}{1zw}
  \item 本時(第4時)の目標
    
    hogehoge

  \item 本時の評価規準
    \begin{itemize}[label=・]
      \item hogehogeしている。[知]
      \item hogehogeしている。[技]
    \end{itemize}

  \item 本時の展開
\end{enumerate}
\begin{longtable}{|c|p{20zw}|p{8zw}|p{8zw}|}
  \hline
  \multicolumn{1}{|c|}{} &
  \multicolumn{1}{c|}{学習内容} &
  \multicolumn{1}{c|}{指導上の留意点} &
  \multicolumn{1}{c|}{評価の観点} 
  \\
  \hline
  \endhead
  %
  導入
  &
  前回までの復習
  &
  &
  \\
  (5分)
  &
  ・こんなことをします。
  &
  ・こんなことに気をつけます。
  &
  ・こんなところを評価します。
  \\
  \hline
  %
  展開\ajMaru{1}
  &
  運動方程式
  &
  &
  \\
  (20分)
  &
  ・こんなことをやります。
  &
  ・こんなことに注意します。
  &
  ・こんな感じで評価します。
  \\
  \hline
  %
  展開\ajMaru{2}
  &
  問題演習
  &
  &
  \\
  (20分)
  &
  ・こんな感じでやる。
  &
  ・hogehoge
  &
  ・hogehoge
  \\
  \hline
  %
  まとめ
  &
  振り返りと次回の予告
  &
  &
  \\
  (5分)
  &
  ・本時の振り返り
  &
  &
  \\
  &
  ・次回の予告
  &
  &
  \\
  \hline
  %
\end{longtable}

  

\end{document}
